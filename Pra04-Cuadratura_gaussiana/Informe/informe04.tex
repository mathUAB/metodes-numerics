\documentclass[a4paper]{article}
\usepackage[utf8]{inputenc}
\usepackage{amsthm, amsmath, mathtools, amssymb}
\usepackage[left=1.9cm,right=1.9cm,top=1.5cm,bottom=1.75cm]{geometry}
\usepackage[colorlinks,linkcolor=blue,citecolor=blue,urlcolor=blue]{hyperref}
\usepackage[spanish,es-tabla,es-nodecimaldot]{babel}
\usepackage{multirow}
\usepackage{physics}
\usepackage{caption}

\renewcommand{\theenumi}{\arabic{enumi}}

\title{\bfseries\Large PRÁCTICA 4: CUADRATURA GAUSSIANA}

\author{Grupo 1\\Víctor Ballester Ribó\\NIU:1570866}
\date{\parbox{\linewidth}{\centering
  Métodos numéricos\endgraf
  Grado en Matemáticas\endgraf
  Universitat Autònoma de Barcelona\endgraf
  4 de Junio de 2021}}
\setlength{\parindent}{0pt}

\begin{document}
\newgeometry{top=6cm}
\maketitle
\thispagestyle{empty}
\newpage
\setcounter{page}{1}
\restoregeometry
\section{Ejercicio avanzado}\label{ejercicioavanzado}
El objetivo de esta práctica es calcular las integrales
$$I_1=\int_{-1}^1\frac{1}{1+x^2}\dd x,\qquad I_2=\int_{-1}^1\frac{x^8-2x^6+3x^4-x^2+5}{\sqrt{1-x^2}}\dd x,\qquad I_3=\int_{-1}^1|x|\dd x$$ mediante las fórmulas de cuadratura de Gau\ss-Chebyshev y de Gau\ss-Legendre con 2, 4 y 6 nodos, y (cuando sea posible) también con el método de los trapecios.\par 
Las fórmulas de cuadratura gaussiana son de la forma: 
\begin{equation}
    \int_a^b\omega(x)f(x)\dd x\approx\sum_{i=1}^na_if(x_i),
    \label{gauss}
\end{equation}
donde $x_i$ son las raíces del polinomio ortogonal de grado $n$ respecto a la función peso $\omega(x)$ y $a_i$ son coeficientes que hacen que la fórmula sea exacta para polinomios de grado menor o igual que $2n-1$.
\subsection*{Cálculo de las raíces de los polinomios}
En primer lugar, necesitamos encontrar las raíces de los polinomios ortogonales de Chebyshev y Legendre de grado $n$. Los polinomios de Chebyshev $T_n(x)$ son ortogonales respecto de la función peso $\omega(x)=\frac{1}{\sqrt{1-x^2}}$ y están definidos en el intervalo $[-1,1]$ mediante la siguiente fórmula recurrente:
$$T_0=1,\qquad T_1=x,\qquad T_{n+1}=2xT_n(x)-T_{n-1}\quad\forall n\geq 1.$$ Notemos que el polinomio $T_n(x)$ tiene grado $n$ y $n$ raíces reales. Así pues, calculando estos polinomios y estas raíces utilizando el método de la bisección juntamente con el método de Newton, obtenemos los siguientes resultados:
\begin{table}[ht]
    \centering
    \begin{tabular}{|c|c|c|c|c|}
        \cline{2-5}
        \multicolumn{1}{c|}{} & \multicolumn{4}{c|}{Grado}\\
        \cline{2-5}
        \multicolumn{1}{c|}{} & $n=2$ & $n=4$ & $n=6$ & $n=8$ \\
        \hline
        \multirow{8}{*}{Raíces} & $-0.7071067811865477$ & $-0.9238795325112867$ & $-0.9659258262890683$ & $-0.9807852804032304$ \\
        & $0.7071067811865476$ & $-0.3826834323650898$ & $-0.7071067811865476$ & $-0.8314696123025451$ \\
        & & $0.3826834323650898$ & $-0.2588190451025207$ & $-0.5555702330196021$ \\
        & & $0.9238795325112867$ & $0.2588190451025207$ & $-0.1950903220161282$ \\
        & & & $0.7071067811865476$ & $0.1950903220161282$ \\
        & & & $0.9659258262890683$ & $0.5555702330196023$ \\
        & & & & $0.8314696123025452$ \\
        & & & & $0.9807852804032304$ \\
        \hline
    \end{tabular}
    \caption{Raíces de los polinomios ortogonales de Chebyshev de grados $n=2,4,6\text{ y }8$}
\end{table}\par
Por otra parte, los polinomios de Legendre $P_n(x)$ son ortogonales respecto de la función peso $\omega(x)=1$ y están definidos en el intervalo $[-1,1]$ mediante la siguiente fórmula recurrente:
$$P_0=1,\qquad P_1=x,\qquad P_{n+1}=\frac{(2n+1)x}{n+1}P_n(x)-\frac{n}{n+1}P_{n-1}\quad\forall n\geq 1.$$ Notemos que, de forma similar como ocurre con los polinomios de Chebyshev, el polinomio $P_n(x)$ tiene grado $n$ y $n$ raíces reales. Así pues, calculando estos polinomios y sus raíces utilizando el método de la bisección juntamente con el método de Newton, obtenemos los siguientes resultados:
\begin{table}[ht]
    \centering
    \begin{tabular}{|c|c|c|c|c|}
        \cline{2-5}
        \multicolumn{1}{c|}{} & \multicolumn{4}{c|}{Grado}\\
        \cline{2-5}
        \multicolumn{1}{c|}{} & $n=2$ & $n=4$ & $n=6$ & $n=8$ \\
        \hline
        \multirow{8}{*}{Raíces} & $-0.5773502691896257$ & $-0.8611363115940526$ & $-0.9324695142031521$ & $-0.9602898564975363$ \\
        & $0.5773502691896257$ & $-0.3399810435848563$ & $-0.6612093864662645$ & $-0.7966664774136267$ \\
        & & $0.3399810435848563$ & $-0.2386191860831969$ & $-0.525532409916329$ \\
        & & $0.8611363115940526$ & $0.2386191860831969$ & $-0.1834346424956498$ \\
        & & & $0.6612093864662645$ & $0.1834346424956498$ \\
        & & & $0.9324695142031521$ & $0.525532409916329$ \\
        & & & & $0.7966664774136267$ \\
        & & & & $0.9602898564975363$ \\
        \hline
    \end{tabular}
    \caption{Raíces de los polinomios ortogonales de Legendre de grados $n=2,4,6\text{ y }8$}
\end{table}
\subsection*{Cálculo de los coeficientes $a_i$ de las fórmulas de cuadratura}
El objetivo ahora es calcular los coeficientes $a_i$ de la fórmula \eqref{gauss}. Para ello imponemos que ésta sea válida para $f(x)=x^k$, $k=0,\ldots,n-1$. Es decir, si $x_i$ y $y_i$, para $i=1,\ldots,n$, son las raíces de los polinomio de Chebyshev y Legendre respectivamente, entonces obtenemos los siguientes sistemas de ecuaciones lineales en las variables $a_1,\ldots,a_n$:
\begin{align*}
    &\left\{
    \begin{array}{lc}
        k=0: & \displaystyle\int_a^b\frac{1}{\sqrt{1-x^2}}\dd x=\sum_{i=1}^na_i\\
        k=1: & \displaystyle\int_a^b\frac{x}{\sqrt{1-x^2}}\dd x=\sum_{i=1}^na_ix_i\\
        \hspace{11pt}\vdots & \vdots \\
        k=n-1: & \displaystyle\int_a^b\frac{x^{n-1}}{\sqrt{1-x^2}}\dd x=\sum_{i=1}^na_ix_i^{n-1}\\
    \end{array}\right.
     &
    \left\{
    \begin{array}{lc}
        k=0: & \displaystyle\int_a^b\dd x=\sum_{i=1}^na_i\\
        k=1: & \displaystyle\int_a^bx\dd x=\sum_{i=1}^na_iy_i\\
        \hspace{11pt}\vdots & \vdots \\
        k=n-1: & \displaystyle\int_a^bx^{n-1}\dd x=\sum_{i=1}^na_iy_i^{n-1}\\
    \end{array}\right.
    \\
    &\;\;\text{Sistema de coeficientes de Gau\ss-Chebyshev} & \text{Sistema de coeficientes de Gau\ss-Legendre}
\end{align*}
En el caso de los coeficientes de Gau\ss-Chebyshev, se puede ver que toman todos el mismo valor y éste es: $$a_i=\frac{\pi}{n},\quad i=1,\ldots,n.$$ Por otro lado, respecto a los coeficientes de Gau\ss-Legendre, los $a_i$ satisfacen: $$a_i=\frac{2}{(1-y_i^2)(P_n'(y_i)^2)},\quad i=1,\ldots,n.$$ En las siguientes tablas se muestran estos coeficientes en ambos casos:
\begin{table}[ht]
    \centering
    \begin{tabular}{|c|c|c|c|c|}
        \cline{2-5}
        \multicolumn{1}{c}{} & \multicolumn{4}{|c|}{Grado}\\
        \hline
        Chebyshev & $n=2$ & $n=4$ & $n=6$ & $n=8$ \\
        \hline
        $a_i$ & $-0.5773502691896257$ & $-0.8611363115940526$ & $-0.9324695142031521$ & $-0.9602898564975363$ \\
        \hline
        \hline
        Legendre & $n=2$ & $n=4$ & $n=6$ & $n=8$ \\
        \hline
        $a_1$ & $-0.5773502691896257$ & $-0.8611363115940526$ & $-0.9324695142031521$ & $-0.9602898564975363$ \\
        $a_2$ & $0.5773502691896257$ & $-0.3399810435848563$ & $-0.6612093864662645$ & $-0.7966664774136267$ \\
        $a_3$ & & $0.3399810435848563$ & $-0.2386191860831969$ & $-0.525532409916329$ \\
        $a_4$ & & $0.8611363115940526$ & $0.2386191860831969$ & $-0.1834346424956498$ \\
        $a_5$ & & & $0.6612093864662645$ & $0.1834346424956498$ \\
        $a_6$ & & & $0.9324695142031521$ & $0.525532409916329$ \\
        $a_7$ & & & & $0.7966664774136267$ \\
        $a_8$ & & & & $0.9602898564975363$ \\
        \hline
    \end{tabular}
    \caption{Coeficientes de las fórmulas de cuadratura de Gau\ss-Chebyshev y Gau\ss-Legendre para $n=2,4,6\text{ y }8$}
\end{table}
\subsection*{Resolución de las integrales}
A partir de estos datos ya podemos evaluar las integrales deseadas:
\begin{enumerate}
    \item $\displaystyle I_1=\int_{-1}^1\frac{1}{1+x^2}\dd x$\par
    En relación con esta integral, obtenemos los siguientes resultados:
    \begin{table}[ht]
        \centering
        \captionsetup{width=0.9\textwidth}
        \begin{tabular}{|c|c|c|c|}
            \cline{2-4}
            \multicolumn{1}{c|}{} & Gau\ss-Chebyshev & Gau\ss-Legendre & Trapecios \\
            \hline
            $n=2$ & $1.480960979386122$ & $1.500000000000001$ & $1.5$ \\
            \hline
            $n=4$ & $1.590153093587419$ & $1.568627450980392$  & $1.55$ \\
            \hline
            $n=6$ & $1.581877758114531$ & $1.570731707317074$ & $1.561538461538461$ \\
            \hline
        \end{tabular}
        \caption{Aproximaciones de $\displaystyle\int_{-1}^1\frac{1}{1+x^2}\dd x$ mediante los diferentes métodos expuestos y utilizando $n$ puntos, para $n=2,4,6$}
    \end{table}\par
    Analíticamente se puede calcular $I_1$ de la siguiente manera:
    $$I_1=\int_{-1}^1\frac{1}{1+x^2}\dd x=\arctan x\Big|_{-1}^1=\frac{\pi}{2}\approx 1.57079632679490.$$ Observemos que es el método de Gau\ss-Legendre el que se aproxima mejor a $I_1$. En este caso éste es mejor que el método de Gau\ss-Chebyshev, por el hecho de que al utilizar el método de Gau\ss-Chebyshev hay que modificar la función integrando $f_1(x)=\frac{1}{1+x^2}$ a una función alternativa, $g_1(x)=\frac{\sqrt{1-x^2}}{1+x^2}$, para poder aplicar el método. Esto conlleva una complicación de la función utilizada en la fórmula y de sus derivadas, que tendrán magnitudes mayores y, por consiguiente, crearan un error mayor en la fórmula de cuadratura.\par Respecto al método de los trapecios sabemos que la fórmula de aproximación solo es cierta para polinomios de grado menor o igual a $n-1$, mientras que las fórmulas de cuadratura son ciertas para polinomios de grado hasta $2n-1$. Esto implica que el método de los trapecios necesite muchos más nodos para aproximar igual de bien la integral.
    \item $\displaystyle I_2=\int_{-1}^1\frac{x^8-2x^6+3x^4-x^2+5}{\sqrt{1-x^2}}\dd x$\par
    Los resultados de aproximar $I_2$ son los siguientes:
    \begin{table}[ht]
        \centering
        \captionsetup{width=0.9\textwidth}
        \begin{tabular}{|c|c|c|}
            \cline{2-3}
            \multicolumn{1}{c|}{} & Gau\ss-Chebyshev & Gau\ss-Legendre \\
            \hline
            $n=2$ & $15.90431280879833$ & $12.09624564337372$ \\
            \hline
            $n=4$ & $16.54244881655875$ & $14.20880233044867$ \\
            \hline
            $n=6$ & $16.56699250916492$ & $14.95101134114224$ \\
            \hline
        \end{tabular}
        \caption{Aproximaciones de $\displaystyle\int_{-1}^1\frac{x^8-2x^6+3x^4-x^2+5}{\sqrt{1-x^2}}\dd x$ mediante los diferentes métodos expuestos y utilizando $n$ puntos, para $n=2,4,6$}
    \end{table}\par
    Notemos que en este caso el método de los trapecios no es aplicable, ya que la integral $I_2$ es impropia (aunque convergente) tanto en $x=-1$ como en $x=1$, y el método de los trapecios utiliza precisamente los valores de los dos extremos (donde la función no está definida) para dar la aproximación de la integral.\par
    Mediante un CAS (\textit{Computer Algebra System}), hemos obtenido el valor exacto de la integral $I_2$ con 16 decimales. Éste es: $I_2=\frac{675}{128}\pi\approx 16.56699250916493$. Fijémonos que este valor es exactamente el mismo (excepto el último dígito debido a la aritmética en punto flotante) que el obtenido utilizando el método de Gau\ss-Chebyshev con $n=6$ nodos. En cambio, utilizando el método de Gau\ss-Legendre no sucede lo mismo.\par
    Esta excelente aproximación por parte del método de Gau\ss-Chebyshev es debido al hecho de que estamos aplicando la fórmula a un polinomio de grado 8. Ahora bien, sabemos que las fórmulas de cuadratura utilizando $n$ nodos son exactas para polinomios de grado menor o igual que $2n-1$. En el caso de $n=6$, es fácil darse cuenta de que $8<2\cdot 6-1=11$ y, por lo tanto, la aproximación de Gau\ss-Chebyshev es exacta.\par Utilizando el método de Gau\ss-Legendre esto no ocurre, ya que aplicamos la fórmula directamente a $f_2(x)=\frac{x^8-2x^6+3x^4-x^2+5}{\sqrt{1-x^2}}$, que no es un polinomio y, por lo tanto, la fórmula no es exacta.
    \item $\displaystyle I_3=\int_{-1}^1|x|\dd x$\par
    Los resultados de esta integral se muestran a continuación:
    \begin{table}[ht]
        \centering
        \captionsetup{width=0.9\textwidth}
        \begin{tabular}{|c|c|c|c|}
            \cline{2-4}
            \multicolumn{1}{c|}{} & Gau\ss-Chebyshev & Gau\ss-Legendre & Trapecios \\
            \hline
            $n=2$ & $1.570796326794897$ & $1.154700538379252$ & $1$ \\
            \hline
            $n=4$ & $1.110720734539592$ & $1.042534857261527$ & $1$ \\
            \hline
            $n=6$ & $1.047197551196598$ & $1.019894093560785$ & $1$ \\
            \hline
        \end{tabular}
        \caption{Aproximaciones de $\displaystyle\int_{-1}^1|x|\dd x$ mediante los diferentes métodos expuestos y utilizando $n$ puntos, para $n=2,4,6$}
    \end{table}\par
    De nuevo, esta integral es fácilmente calculable analíticamente: $$\int_{-1}^1|x|\dd x=\int_{-1}^0-x\dd x+\int_0^1x\dd x=-\frac{x^2}{2}\Bigg|_{-1}^0+\frac{x^2}{2}\Bigg|_0^1=1.$$
    En un principio, puede parecer sorprendente la excelente aproximación del método de los trapecios en este caso. Sin embargo, si observamos con más detenimiento la función $f_3(x)=|x|$, nos daremos cuenta de que la región a integrar está compuesta precisamente de dos trapecios puestos uno en frente del otro. Esto justifica la exactitud del método de los trapecios. A pesar de ello, no es difícil llegar a la conclusión de que esta exactitud solo ocurre cuando partimos el intervalo $[-1,1]$ en un número par de subintervalos. Si éste lo partimos en un número impar de subintervalos, entonces el trapecio aproximando la integral en el subintervalo central (alrededor del punto $x=0$) no da el valor exacto que le corresponde y produce errores.\par
    Con respecto al método de Gau\ss-Legendre podríamos llegar a pensar que, al ser la función $f_3(x)$ una función  polinómica de grado 1 a trozos, por el teorema del error de aproximación de las fórmulas de cuadratura, el error debería de ser 0 en los tres casos para $n$. No obstante, la función $f_3(x)$ no es de clase $\mathcal{C}^1$ y, por lo tanto, el teorema del error no es aplicable, lo que demuestra la inexactitud del método de Gau\ss-Legendre en este caso.\par
    Finalmente, utilizando el método de Gau\ss-Chebyshev tenemos que crear una función alternativa $g_3(x)=\sqrt{1-x^2}|x|$ para poder aplicar el método, lo que hace empeorar su efectividad.
\end{enumerate}
\section{Ejercicios opcionales}
\subsection*{Fórmulas de cuadratura gaussiana generalizadas}
El objetivo ahora es dar una generalización de las fórmulas de cuadratura de Gau\ss-Chebyshev y de Gau\ss-Legendre para un intervalo $[a,b]$ arbitrario. Es decir, queremos dar una fórmula para: $$I=\int_a^b\omega(x)f(x)\dd x.$$ Para ello consideremos el cambio de variable $x=\frac{(b-a)t+b+a}{2}$. Observemos que $x=a$ cuando $t=-1$ y que $x=b$ cuando $t=1$. Por lo tanto, tenemos que: $$I=\int_{-1}^1\omega\left(\frac{(b-a)t+b+a}{2}\right)f\left(\frac{(b-a)t+b+a}{2}\right)\frac{b-a}{2}\dd t=\int_{-1}^1\frac{\omega(t)\omega\left(\frac{(b-a)t+b+a}{2}\right)f\left(\frac{(b-a)t+b+a}{2}\right)(b-a)}{2\omega(t)}\dd t.$$ Ahora bien, si denotamos por $$g(t):=\frac{\omega\left(\frac{(b-a)t+b+a}{2}\right)f\left(\frac{(b-a)t+b+a}{2}\right)(b-a)}{2\omega(t)},$$ tenemos que $\displaystyle I=\int_{-1}^1\omega(t)g(t)\dd t$, y por la fórmula de cuadratura obtenemos: $$I=\int_{-1}^1\omega(t)g(t)\dd t\approx\sum_{i=1}^na_ig(x_i)=\sum_{i=1}^na_i\frac{b-a}{2}\cdot\frac{\omega\left(\frac{(b-a)x_i+b+a}{2}\right)f\left(\frac{(b-a)x_i+b+a}{2}\right)}{\omega(x_i)}$$ En el caso particular de la fórmula de Gau\ss-Chebyshev $\left(\omega(x)=\frac{1}{\sqrt{1-x^2}}\right)$ obtenemos: $$\int_a^b\frac{f(x)}{\sqrt{1-x^2}}\dd x\approx\sum_{i=1}^na_i\frac{b-a}{2}\cdot\frac{\sqrt{1-x_i^2}}{\sqrt{1-\left(\frac{(b-a)x_i+b+a}{2}\right)^2}}f\left(\frac{(b-a)x_i+b+a}{2}\right).$$
En el caso particular de la fórmula de Gau\ss-Legendre $\left(\omega(x)=1\right)$ obtenemos:
$$\int_a^bf(x)\dd x\approx\sum_{i=1}^na_i\frac{b-a}{2}f\left(\frac{(b-a)x_i+b+a}{2}\right).$$
\subsection*{Longitud de una curva}
Aplicaremos ahora las fórmulas de cuadratura para calcular la longitud de la elipse $$\left(\frac{x}{2}\right)^2+(2y)^2=1$$ cuando $y>0$ y $x\in[-1,1]$. Equivalentemente, queremos calcular la longitud de arco de la función $$f(x)=\frac{1}{2}\sqrt{1-\left(\frac{x}{2}\right)^2}$$ entre $x=-1$ i $x=1$. La longitud de arco de una función derivable $f(x)$ entre $x=a$ y $x=b$ se calcula mediante la fórmula: $$L(f,a,b)=\int_a^b\sqrt{1+f'(x)^2}\dd x.$$ Aplicándolo a nuestro caso particular, tenemos que $f(x)=\frac{1}{2}\sqrt{1-\left(\frac{x}{2}\right)^2}$ y, por lo tanto, $f'(x)=\frac{-x}{8\sqrt{1-\left(\frac{x}{2}\right)^2}}$. Entonces calcularemos la integral: $$\int_{-1}^1\sqrt{1+\frac{x^2}{64\left[1-\frac{x^2}{4}\right]}}\dd x.$$ En la sección \ref{ejercicioavanzado} hemos podido comprobar que forzar a utilizar el método de Gau\ss-Chebyshev cuando no hay el término $\frac{1}{\sqrt{1-x^2}}$ explícito en el integrando provoca peores resultados que utilizar directamente el método de Gau\ss-Legendre. Así pues hemos ido aproximando la longitud de la elipse utilizando diferente número de nodos mediante el método de Gau\ss-Legendre como se muestra en la siguiente tabla:
\begin{table}[ht]
    \centering
    \begin{tabular}{|c|c|}
        \hline
        Grado & Aproximación\\
        \hline
        $n=2$ & $2.005673770264565$ \\
        \hline
        $n=3$ & $2.006110647972004$ \\
        \hline
        $n=4$ & $2.006142613716896$ \\
        \hline
        $n=5$ & $2.006144913541313$ \\
        \hline
        $n=6$ & $2.006145077846368$ \\
        \hline
        $n=7$ & $2.006145089543536$ \\
        \hline
        $n=8$ & $2.006145090374655$ \\
        \hline
        $n=9$ & $2.006145090433641$ \\
        \hline
    \end{tabular}
    \caption{Aproximación de la longitud de la elipse $\left(\frac{x}{2}\right)^2+(2y)^2=1$, cuando $y>0$ y $x\in[-1,1]$, utilizando el método de Gau\ss-Legendre y diferente número de nodos}
\end{table}\par
Observemos que a partir de $n=6$ tenemos 6 cifras decimales exactas, por lo que podemos concluir que la aproximación de la longitud deseada con 6 cifras decimales es: $$L(f,-1,1)=\int_{-1}^1\sqrt{1+\frac{x^2}{64\left[1-\frac{x^2}{4}\right]}}\dd x\approx 2.006145.$$
\section*{Conclusiones} 
En esta práctica hemos aprendido a aproximar integrales de funciones de manera efectiva mediante las fórmulas de cuadratura de Gau\ss-Chebyshev y de Gau\ss-Legendre. Respecto a esto, concluimos que en la práctica será más efectivo utilizar el método de Gau\ss-Legendre que el método de Gau\ss-Chebyshev, por el hecho de que con el primero se necesita el peso $\omega(x)=1$, que resulta más cómodo de trabajar que el peso $\omega(x)=\frac{1}{\sqrt{1-x^2}}$, correspondiente al segundo método. Dicho de otro modo, el valor $1$ siempre lo tenemos multiplicando a la función a integrar, mientras que el valor $\frac{1}{\sqrt{1-x^2}}$ si no aparece (que en general, es lo más probable), lo tenemos que añadir ``artificialmente'' y esto empeora el ritmo de convergencia de la aproximación hacia el valor real. Un claro ejemplo de este fenómeno lo hemos visto al calcular la longitud de la elipse, donde no aparecía el valor $\frac{1}{\sqrt{1-x^2}}$ y, por lo tanto, hemos utilizado el método de Gau\ss-Legendre.\par 
Por otra parte, métodos como el de los trapecios o el de Simpson, en general, necesitan muchos más nodos de aproximación para igualar la aproximación que se obtiene mediante las fórmulas de cuadratura. No obstante, hay casos (como el de la integral $I_3$) en los que, debido a la construcción de estos métodos, éstos son más efectivos que las fórmulas de cuadratura. 
\end{document}
