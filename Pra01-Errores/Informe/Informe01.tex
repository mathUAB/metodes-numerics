\documentclass[a4paper]{article}
\usepackage[utf8]{inputenc}
\usepackage{amsthm, amsmath, mathtools, amssymb}
\usepackage[left=2cm,right=2cm,top=1.9cm,bottom=2cm]{geometry}
\usepackage[colorlinks,linkcolor=blue,citecolor=blue,urlcolor=blue]{hyperref}
\usepackage[spanish,es-tabla,es-nodecimaldot]{babel}
\usepackage{multirow}

\renewcommand{\labelenumii}{\alph{enumii})}
\title{\bfseries\Large PRÁCTICA 1: ERRORES}

\author{Grupo 1\\Júlia Albero Pes, NIU:1566550\\Víctor Ballester Ribó, NIU:1570866}
\date{\parbox{\linewidth}{\centering
  Métodos numéricos\endgraf
  Grado en Matemáticas\endgraf
  Universitat Autònoma de Barcelona\endgraf
  15 de Marzo de 2021}}
%\setlength{\parindent}{0pt}

\begin{document}
\newgeometry{top=6cm}
\maketitle
\thispagestyle{empty}
\newpage
\setcounter{page}{1}
\restoregeometry
\section*{Problema 1}
Los resultados que hemos obtenido al evaluar la función \begin{equation}
    f(x)=\left\{\begin{array}{ccc}
        \cfrac{1-\cos(x)}{x^2} & \text{si} & x\ne 0 \\
        \cfrac{1}{2} & \text{si} & x=0 
    \end{array}\right.
    \label{f}
\end{equation} en el punto $x_0=1.2\times10^{-5}$ en precisión simple y en precisión doble se muestran en la tabla siguiente:\par
\begin{table}[ht]
    \centering
    \begin{tabular}{|c|c|c|}
        \hline
         & Valor en precisión simple & Valor en precisión doble \\
         \hline
         $f(x_0)$ & 0 & 0.4999997329749008 \\
         \hline
    \end{tabular}
    \caption{Resultados obtenidos al evaluar $f(x)$ en el punto $x_0=1.2\times10^{-5}$ con simple y doble precisión.}
    \label{tab:1}
\end{table}
Vemos claramente que el programa en precisión simple está fallando. El problema surge al evaluar la función $\cos(x)$ en un número relativamente cercano a 0. Sabemos que un número guardado en precisión simple tiene 7 u 8 cifras significativas. Ahora bien, si calculamos el valor de $\cos(x_0)$ con al menos 12 cifras exactas, por ejemplo, tenemos que $\cos(x_0)\approx 0.999999999928$. Pero inmediatamente vemos que este número aproximado a 8 cifras significativas es 1, por lo que el ordenador, en precisión simple, asigna $\cos(x_0)=1$. Y por lo tanto: 
\begin{equation*}
    f(x_0)=\frac{1-\cos(x_0)}{x_0^2}\approx\frac{1-1}{x_0^2}=0.
\end{equation*}
De aquí sale el 0 obtenido al evaluar la función. Con precisión doble este fenómeno no pasa porque un número guardado en precisión doble tiene 15 o 16 cifras significativas, con lo cual, viendo la expresión anterior de $\cos(x_0)$ con 12 cifras, se intuye que en doble precisión $\cos(x_0)\ne 1$.\par
Para reducir el error utilizando la expresión \eqref{f}, emplearemos la relación trigonométrica siguiente: $$1-\cos(x)=2\sin(x/2)^2.$$ Substituyendo esta expresión en \eqref{f} obtenemos la nueva función:
\begin{equation}
g(x)=\left\{\begin{array}{ccc}
    \cfrac{2\sin(x/2)^2}{x^2} & \text{si} & x\ne 0 \\
    \cfrac{1}{2} & \text{si} & x=0 
\end{array}\right.
\label{g}
\end{equation}
que es matemáticamente equivalente a $f(x)$.\par Los resultados obtenidos al evaluar $g(x)$ en $x=x_0$ se muestran en la tabla siguiente:\par
\begin{table}[ht]
    \centering
    \begin{tabular}{|c|c|c|}
        \hline
         & Algoritmo en precisión simple & Algoritmo en precisión doble \\
         \hline
         $f(x_0)$ & $0.5$ & $0.499999999994$\\
         \hline
    \end{tabular}
    \caption{Resultados obtenidos al evaluar $g(x)$ en el punto $x_0=1.2\times10^{-5}$ con simple y doble precisión.}
    \label{tab:2}
\end{table}
Observando las tablas \ref{tab:1} i \ref{tab:2} vemos una clara diferencia entre los resultados, especialmente en precisión simple. La razón por la que $g(x_0)=0.5\ne0$ es la siguiente. Fijémonos que $\sin(x_0/2)\approx5.99999999996400\times 10^{-6}$. Por lo tanto, trabajando en precisión simple, tenemos que $\sin(x_0/2)=6\times 10^{-6}$. Haciendo los cálculos obtenemos:
$$\frac{2\times\sin(x_0/2)\times\sin(x_0/2)}{x_0\times x_0}=\frac{2\times 6\times 10^{-6}\times 6\times 10^{-6}}{1.2\times 10^{-5}\times 1.2\times 10^{-5}}=\frac{7.2\times 10^{-11}}{1.44\times 10^{-10}}=0.5.$$
En este caso no tenemos error de cancelación como sí ocurría en el caso de $f(x_0)$.\par
Cabe señalar que también obtenemos una diferencia, aunque más pequeña, en los valores de $f(x_0)$ y $g(x_0)$ en precisión doble. Esto también se debe al mismo factor de cancelación que hemos comentado. En efecto, tenemos que $\cos(x_0)=0.9999999999280000$, en precisión doble. Es importante mencionar que a partir del último 0, los números que aparezcan no serán relativos de la propia función, ya que en precisión doble trabajamos solo con 16 cifras significativas. Por este motivo, cuando restamos este último número a 1 obtenemos solamente las primeras cifras correctas, mientras que las demás son inciertas. Por lo tanto, una vez dividido por $x^2$, obtenemos los primeros dígitos correctos ($0.499999$) y los que siguen, incorrectos. Observamos que utilizando la función $g(x)$ este problema no ocurre.
\newpage
\section*{Problema 2}
La solución de una ecuación cuadrática con coeficientes reales $$ax^2+bx+c=0,\quad a\ne0$$ se obtiene a partir de la expresión 
\begin{equation}
    x=\frac{-b\pm\sqrt{b^2-4ac}}{2a}.
    \label{2}
\end{equation}
Suponiendo que $a>0$ y $b^2>4ac$, como se menciona en el enunciado, podríamos tener errores de cancelación en una de las soluciones cuando $b^2\gg 4ac$. En efecto, si $b^2\gg4ac$ tenemos que $\frac{4ac}{b^2}\ll1$. Ahora bien, si $\frac{4ac}{b^2}$ es del orden de $10^{-8}$, significa que $b^2$ es del orden de $10^8$ veces más grande que $4ac$. En precisión simple esto se traduce a la igualdad $b^2-4ac=b^2$, ya que la primera cifra de $b^2$ y la primera cifra de $4ac$ están separadas por más de 8 posiciones (que es el numero de cifras significativas con el que trabajamos en precisión simple) en el valor $b^2-4ac$. Es por esto que una de las soluciones sería 0.
\begin{equation}
x=\frac{-b\pm\sqrt{b^2-4ac}}{2a}\approx\frac{-b\pm|b|}{2a}=\left\{\begin{array}{ccccc} \cfrac{-b+b}{2a}=0 & \text{y} & \cfrac{-b-b}{2a}=\cfrac{-b}{a} & \text{si} & b>0\\\\ \cfrac{-b+(-b)}{2a}=\cfrac{-b}{a} & \text{y} & \cfrac{-b-(-b)}{2a}=0 & \text{si} & b<0
\end{array}\right.
\end{equation}
Sin embargo, podemos prescindir de este error multiplicando y dividiendo la solución que sufre el error de cancelación por el conjugado del numerador de la fracción. Para el caso $b>0$, tenemos que:
\begin{equation}
\frac{(-b+\sqrt{b^2-4ac})}{2a}\cdot\frac{(-b-\sqrt{b^2-4ac})}{(-b-\sqrt{b^2-4ac})}=\frac{b^2-b^2+4ac}{2a\cdot(-b-\sqrt{b^2-4ac})}=\frac{4ac}{2a\cdot(-b-\sqrt{b^2-4ac})}=\frac{-2c}{b+\sqrt{b^2-4ac}}.	
\label{conju1}
\end{equation} 
Notemos que para valores $b^2\gg4ac$ tendremos $\displaystyle\frac{-2c}{b+\sqrt{b^2-4ac}}\approx \frac{-c}{b}$.\par Si $b<0$, procedemos de forma análoga a \eqref{conju1}.
\begin{equation}
\frac{(-b-\sqrt{b^2-4ac})}{2a}\cdot\frac{(-b+\sqrt{b^2-4ac})}{(-b+\sqrt{b^2-4ac})}=\frac{b^2-b^2+4ac}{2a\cdot(-b+\sqrt{b^2-4ac})}=\frac{4ac}{2a\cdot(-b+\sqrt{b^2-4ac})}=\frac{-2c}{b-\sqrt{b^2-4ac}}	.
\label{conju2}
\end{equation}
Veamos ahora un ejemplo de cómo se comportan estos dos algoritmos para calcular las raíces del polinomio $x^2+100000x+3$ en precisión simple y doble. Los resultados se exponen en la siguiente tabla:\par
\begin{table}[ht]
	\centering
	\begin{tabular}{|c|c|c||c|c|}
	    \hline
	    & \multicolumn{2}{c||}{Precisión simple} & \multicolumn{2}{c|}{Precisión doble}\\
		\cline{2-5} 
		& Solución 1 & Solución 2 & Solución 1 & Solución 2\\
		\hline
		Método ordinario & $0$ & $-100000$ & $-3.000000288011506\times10^{-5}$ & $-99999.99997$ \\
		\hline
		Método propuesto &  $-2.9999999\times10^{-5}$ & $-100000$ & $-3.0000000009\times10^{-5}$ & $-99999.99997$\\
		\hline
	\end{tabular}
		\caption{Soluciones del polinomio $x^2+100000x+3$ mediante los dos métodos expuestos}
		\label{tab:10}
	\end{table}
En este caso, en precisión simple tenemos que: $$b^2-4ac=100000^2-4\cdot 3=10^{10}-12\approx10^{10}.$$
Y por consiguiente: $$\frac{-b+\sqrt{b^2-4ac}}{2a}\approx\frac{-100000+\sqrt{10^{10}}}{2}=\frac{-100000+100000}{2}=0.$$ Sin embargo, mediante el método propuesto este valor se convierte en $-2.9999999\times 10^{-5}$, que tiene un error mucho menor que el calculado con el método ordinario.\par Por otro lado, observamos que de forma similar al problema 1, en precisión doble también obtenemos cierta incertidumbre en los últimos dígitos de la solución 1 calculada con el método ordinario. Esto se debe al mismo error de cancelación con el cual obtenemos la solución 0 en precisión simple.
\newpage
\section*{Problema 3}
En estadística la varianza muestral de $n$ números se define como \begin{equation}
    s_{n}^2=\frac{1}{n-1}\sum_{i=1}^{n}(x_i-\bar{x}),\quad\text{donde}\quad\frac{1}{n}\sum_{i=1}^{n}x_{i}.
\end{equation}
O alternativamente como:  
\begin{equation}
    s_{n}^{2}=\frac{1}{n-1}\left[\sum_{i=1}^{n}x_{i}^2-\frac{1}{n}\left(\sum_{i=1}^{n}x_{i}\right)^2\right].
    \label{v2}
\end{equation} Denotaremos estos dos métodos para calcular la varianza muestral como $V_{1}$ y $V_{2}$, respectivamente. Seguidamente mostraremos cómo se comportan las fórmulas de $V_1$ y $V_2$ en varios ejemplos de muestras y en las distintas precisiones.\par Para $n=3$, consideramos el vector $x=\{10000,10001,10002\}$. Los resultados de los cálculos se muestran en la tabla siguiente:\par
\begin{table}[ht]
	\centering
	\begin{tabular}{|c|c|c|c|}
	    \hline
		& Precisión simple & Precisión doble \\
		\hline 
		$V_{1}$ & 1 & 1\\
		\hline
		$V_{2}$ & 0 & 1\\
		\hline
	\end{tabular}
		\caption{Comportamiento de la varianza muestral mediante las diferentes fórmulas}
		\label{tab:11}
\end{table}
Observamos que mientras la fórmula $V_{1}$ en precisión simple da 1, $V_{2}$ da 0. Esto se debe al hecho de que el cálculo en la segunda fórmula utiliza valores de orden de $10^{8}$ (procedentes de $10000^2$) en donde pueden existir valores no nulos en las cifras menos significativas. Concretamente, si hacemos el cálculo de $\frac{1}{n}\left(\sum_{i=1}^{n}x_{i}\right)^2$ de la fórmula \eqref{v2} con los datos propuestos obtenemos este resultado:
\begin{equation}
    \frac{1}{3}\left(\sum_{i=1}^3x_{i}\right)^2=\frac{1}{3}(10000+10001+10002)^2=300060003.
    \label{v11}
\end{equation}
Calculemos ahora $\sum_{i=1}^{n}x_{i}^2$: 
\begin{equation}
\sum_{i=1}^{3}x_{i}^2=10000^2+10001^2+10002^2=300060005.
\label{v12}
\end{equation}
Fijémonos que los valores de \eqref{v11} y \eqref{v12} son valores iguales en precisión simple ya que solo cambia la novena cifra significativa. Es decir, el programa los lee como $3.0006\times10^8$ en ambos casos, lo que conlleva que al restarlos obtengamos 0.\par
Esto no ocurre en $V_{1}$ ya que cuando elevamos al cuadrado, que es la operación ``peligrosa", ya hemos restado previamente los números, reduciendo así su orden de magnitud. De esta forma, el programa no tiene problemas de cancelación. Es decir como $\bar{x}=\frac{1}{3}\sum_{i=1}^{n}x_{i}=10001$, entonces $$V_{1}=\frac{1}{2}\sum_{i=1}^{3}(x_i-\bar{x})=\frac{1}{2}\left[(10000-10001)^2+(10001-10001)^2+(10002-10001)^2\right]=\frac{1}{2}\left[1+0+1\right]=1.$$
Veamos mejor estas discrepancias usando los siguientes vectores de 200 componentes: $$x_{1}=\{1000000,1000001,\dots,1000199\}\quad\text{y}\quad x_{2}=\{1, 1.000001, 1.000002, \dots, 1.000199\}.$$\par
\begin{table}[ht]
	\centering
	\begin{tabular}{|c|c|c||c|c|}
	    \hline
	    & \multicolumn{2}{c||}{Precisión simple} & \multicolumn{2}{c|}{Precisión doble}\\
		\cline{2-5} 
		& Vector $x_{1}$ & Vector $x_{2}$ & Vector $x_{1}$ & Vector $x_{2}$\\
		\hline
		$V_{1}$ & $3350.1887$ & $3.3502197\times 10^{-9}$ & $3350$ & $3.350000000000287\times 10^{-9}$\\
		\hline
		$V_{2}$ & $843076.19$ & $-1.5335466\times 10^{-7}$ & $3350$ & $3.350000478024609\times 10^{-9}$\\
		\hline
	\end{tabular}
		\caption{Discrepancias de la varianza muestral mediante los vectores $x_{1}$ y $x_{2}$}
		\label{tab:12}
	\end{table}
Observamos que en el caso del vector $x_1$ utilizando $V_2$ obtenemos un valor mucho más grande del que deberíamos. De nuevo, esto se debe a la pérdida de números significativos al elevar al cuadrado los términos de la fórmula $V_2$. Un caso más interesante es el que ocurre con el vector $x_2$. Resulta que aplicando el método $V_2$ obtenemos una varianza negativa, que no es posible de ninguna manera. En efecto, si calculamos la suma $\frac{1}{n}\left(\sum_{i=1}^{n}x_{i}\right)^2$ aplicada al vector $x_2$ obtendremos, en precisión simple, el valor de $200.03986$. Por otro lado, el valor de $\sum_{i=1}^{n}x_{i}^2$ es $200.03983$, de donde se desprende que la diferencia entre ambos sea negativa. Esto se debe, una vez más, a la pérdida de cifras significativas al evaluar los números al cuadrado.
\newpage
\section*{Problema 4}
En la tabla siguiente se muestra el valor de las sumas parciales $S_N$ de la serie numérica 
\begin{equation}
    S=\sum_{k=1}^\infty\frac{1}{k^2},
    \label{pi}
\end{equation} calculadas en precisión simple y doble, y en orden creciente y decreciente para varios valores de $N$.\par
\begin{table}[ht]
	\centering
	\begin{tabular}{|c|c|c|c|}
	    \hline
	    & \multicolumn{3}{c|}{Valor de $S_N$ en precisión simple} \\
		\cline{2-4} 
		& $N=5000$ & $N=10000$ & $N=30000$\\
		\hline
		Orden creciente  & $1.6447253$ & $1.6447253$ & $1.6447253$ \\
		\hline
		Orden decreciente  & $1.6447341$ & $1.6448340$  & $1.6449008$ \\
		\hline
	\end{tabular}\vspace{5pt}
	\begin{tabular}{|c|c|c|c|}
	\hline
	    & \multicolumn{3}{c|}{Valor de $S_N$ en precisión doble}\\
		\cline{2-4} 
		& $N=5000$ & $N=10000$ & $N=30000$ \\
		\hline
		Orden creciente & $1.644734086846901$ & $1.644834071848065$ & $1.644900734070444$ \\
		\hline
		Orden decreciente & $1.644734086846893$ & $1.644834071848060$ & $1.644900734070442$\\
		\hline
	\end{tabular}
		\caption{Cálculo de la suma $S_N$ para varios valores de $N$ en simple y doble precisión}
		\label{tab:4}
	\end{table}
El valor exacto de la suma de \eqref{pi} sabemos que es $S=\pi^2/6=1.644934066848226...$ De nuevo, en precisión doble no tenemos nada que destacar; ambas sumas (en orden creciente y decreciente) coinciden, excepto en los últimos dígitos, donde pueden variar varios dígitos debido a las aproximaciones de los números en punto flotante. El problema está al sumar la serie utilizando precisión simple. Fijémonos que, sumando la serie en orden ascendente, un vez llegado al término 5000 de la suma tenemos que realizar la siguiente operación: $$S_{5000}=S_{4999}+\frac{1}{5000^2}=1.64...+4\times10^{-8}.$$ Ahora bien, como trabajamos en precisión simple, esta última suma da exactamente $S_{4999}$, ya que solo se almacenan 8 dígitos significativos y justamente el dígito 4 de $4\times10^{-8}$ está en la posición de la novena cifra significativa en la suma $1.64...+4\times10^{-8}$. Claramente también ocurrirá este error al sumar los términos $5001,5002,\ldots,30000$. Por este motivo el valor de $S_N$ en precisión simple y orden creciente para $N=5000$, $N=10000$ y $N=30000$ da exactamente el mismo resultado.\par 
Por otro lado, si hacemos la suma en orden decreciente, este error de aproximación no se percibe por el siguiente motivo. Como los primeros dígitos de la suma son $1/N^2$, $1/(N-1)^{2},\ldots$ y éstos tienen la misma magnitud, no perdemos los dígitos más significativos de estos valores que sí perdíamos sumando la serie en orden creciente. A medida que vayamos sumando términos más grandes, iremos perdiendo los últimos dígitos significativos del valor de la suma acumulada. Pero estas pérdidas las haremos de forma ``progresiva" de manera que no se percibirá tanto el error final. Con el método anterior, esta ``progresividad" no la teníamos, ya que a partir de un cierto número las sumas parciales permanecían constantes.\par
Para calcular una fórmula alternativa que se comporte mejor que \eqref{pi} utilizaremos la serie de potencias de la función $\arcsin{x}$: $$\arcsin{x}=\sum_{k=0}^\infty\frac{(2k)!}{2^{2k}(k!)^2}\frac{x^{2k+1}}{2k+1}\quad\text{para }|x|\leq 1.$$ Notemos que $\arcsin{(1/2)}=\pi/6$ y por consiguiente tenemos que: 
\begin{equation}
\frac{\pi}{6}=\sum_{k=0}^\infty\frac{(2k)!}{2^{2k}(k!)^2}\frac{1}{2k+1}\frac{1}{2^{2k+1}}\implies\frac{\pi^2}{6}=6\left(\sum_{k=0}^\infty\frac{(2k)!}{2^{2k}(k!)^2}\frac{1}{2k+1}\frac{1}{2^{2k+1}}\right)^2.
\label{r}
\end{equation}
Denotamos $R_N$, el valor de la suma parcial $N$-ésima de la serie de potencias de $\arcsin{x}$. En la siguiente tabla se muestran las aproximaciones parciales de $\pi^2/6$ obtenidas a partir de la fórmula \eqref{r} haciendo la suma de $R_N$ en orden decreciente.\par
\begin{table}[ht]
	\centering
	\begin{tabular}{|c|c|c|c|}
	    \hline
		& $N=4$ & $N=9$ & $N=21$\\
		\hline
		Valor de $6\cdot (R_N)^2$ en precisión simple & $1.6448488$ & $1.6449342$ & $1.6449342$\\
		\hline
		Valor de $6\cdot (R_N)^2$ en precisión doble & $1.644848740989538$ & $1.644934034679181$ & $1.644934066848226$\\
		\hline
	\end{tabular}
		\caption{Cálculo de la suma $6\cdot (R_N)^2$ para varios valores de $N$ en simple y doble precisión}
	\end{table}
Notemos la diferencia de velocidad de convergencia de este método en comparación con el anterior. En particular, en precisión doble, con solamente 21 términos somos capaces de conseguir 16 cifras (el máximo posible) exactas de $\pi^2/6$ mientras que con el método anterior y utilizando 30000 términos solo éramos capaces de calcular 5 cifras exactas de este mismo valor. 
\newpage
\section*{Conclusiones} 
La conclusión principal de esta práctica es que los números guardados en punto flotante no son exactos. El ordenador tiene una memoria finita y es por este motivo que tiene que decidir cuántos dígitos significativos guardar, según en qué precisión se esté trabajando. Este motivo es la causa principal de muchos errores observados a lo largo de la práctica.\par Entre ellos está el de no poder sumar (o restar) un número muy grande con un número muy pequeño, ya que si la primera cifra significativa del primer número y la primera cifra significativa del segundo están separadas por más de 8 posiciones (trabajando en precisión simple), entonces el valor de la suma será el mismo que el número mayor. Éste es uno de los varios casos con los que nos hemos ido encontrando a lo largo de esta práctica.\par Otro caso peculiar y sorprendente es que el problema 4 nos ha evidenciado que las operaciones en punto flotante no son asociativas. En efecto, empezando a sumar los términos de una serie parcial desde el inicio hemos obtenido diferentes resultados que empezando a sumar los términos desde el final.\par En definitiva, observando los resultados de cada problema, podemos concluir que con precisión doble estos errores no ocurren tan a menudo, al menos en la magnitud trabajada a lo largo de la práctica. Es por esto que concluimos que hay que trabajar siempre, si es posible, en precisión doble para evitar estos errores, que a veces pueden pasar desapercibidos.
\end{document}
